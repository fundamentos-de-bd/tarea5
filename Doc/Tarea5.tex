\documentclass{article}
\usepackage[utf8]{inputenc}
\usepackage[spanish]{babel}
\usepackage{amssymb}

% Formato de página
\usepackage[letterpaper, margin = 1.5cm]{geometry}

% Más opciones para enumerar
\usepackage{enumitem}

% Manejo de imágenes
\usepackage{graphicx}
\usepackage{wrapfig}
\graphicspath{{img/}}
\usepackage{float}

\begin{document}
    \title{
        Fundamentos de bases de datos \\
        Tarea 5 \\
        Normalización
    }
    \author{
        Díaz Gómez Silvia \\
        Eugenio Aceves Narciso Isaac \\
        Quiroz Castañeda Edgar
    }
    \date {
        26 de Abril del 2019    
    }
    \maketitle
    
    \begin{enumerate}
    	
    
    \item { \textbf{Preguntas de repaso}}
    \begin{enumerate}[label = \alph*.]
        \item ¿Qué es una dependencia funcional y cómo se define?
        \item ¿Para qué sirve el concepto de \textbf{dependencia} en la normalización?
        \item Sea A la llave de R(A, B, C). Indica \textbf{todas} las dependencias funcionales que implica \textbf{A}. \\
        $A \rightarrow BC$\\
        
        \item ¿Qué es una forma normal? ¿Cuál es el objetivo de normalizar un modelo de datos?
        \item ¿En qué casos es preferible lograr \textbf{3NF} en vez de \textbf{BCNF}?
    \end{enumerate}

    \item
    Proporciona algunos ejemplos que demuestren que las siguientes reglas no son válidas:
    \begin{enumerate}
    	\item Si \textbf{A $\rightarrow$ B}, entonces  \textbf{B $\rightarrow$ A}
    	\item Si \textbf{AB $\rightarrow$ C}, entonces  \textbf{A $\rightarrow$ C y B $\rightarrow$ C}
    	\item Si \textbf{A $\twoheadrightarrow$ C}, entonces \textbf{A $\rightarrow$ C}
    \end{enumerate}
       
    \item Para cada uno de los esquemas que se muestran a continuación:
    
    \begin{enumerate}
    	\item \textbf{R(A,B,C,D,E)} con \textbf{F=\{AB $\rightarrow$ CD, E $\rightarrow$ C, D $\rightarrow$ B \}}
    	\item \textbf{R(A,B,C,D,E)}  con \textbf{F=\{AB $\rightarrow$ C, DE $\rightarrow$ C, B $\rightarrow$ D \}}
    \end{enumerate}   

\begin{itemize}
	\item Especifica  de ser posible \textbf{dos DF no triviales} que se puedan derivar de las dependencias funcionales dadas.
	Usando las reglas de inferencia de Armstrong podemos deducir algunas DF.
	\begin{enumerate}
		\item \textbf{Aumento:} DC $\rightarrow$ BC y EA $\rightarrow$ CA\\ 
		\item 
		 
		
		
	\end{enumerate}
	\item Indica una \textbf{llave candidata} para \textbf{R} \\
	Usamos la cerradura para encontrar una llave candidata
	\begin{enumerate}
		\item \{AB\}+= \{ABCD\}, \{E\}+= \{EC\}, \{D\}+= \{DB\}, la cerradura de AB es la que contiene mas elementos de la relación por lo tanto una llave para R sería : \textbf{ABE}\\ 
		\item \{AB\}+= \{ABCD\}, \{DE\}+= \{E\}, \{B\}+= \{BD\}, la cerradura de AB es la que contiene mas elementos de la relación por lo tanto una llave para R sería : \textbf{ABE}\\ 
		
		
		
	\end{enumerate}
	
	\item Especifica \textbf{todas las violaciones}  a la \textbf{BCNF}\\
	\begin{enumerate}
		\item Las tres dependencias son violaciones a BCNF porque no aparece del lado izquierdo de las DF la llave que es ABE.\\
		
		\item Las tres dependencias son violaciones a BCNF porque no aparece del lado izquierdo de las DF la llave que es ABE.
	\end{enumerate}
	\item \textbf{Normaliza} de acuerdo a \textbf{BCNF}, asegúrate de indicar cuáles son las relaciones resultantes con sus respectivas dependencias funcionales.
	\begin{enumerate}
		\item Como todas las DF son violaciones, tomamos a AB $\rightarrow$ CD y su cerradura es  \{AB\}+= \{ABCD\}.
		Así que definimos dos nuevas relaciones S y T,\\
		S(A,B,C,D) con \{AB $\rightarrow$ B , D $\rightarrow$ B\}\\
		T(A,B,E) con ABE $\rightarrow$ ABE y perdemos E $\rightarrow$ C\\
		En S la llave es AB por lo tanto D $\rightarrow$ B es una violación.\\
		Ahora para S tomamos D $\rightarrow$ B y calculamos la cerradura para \{D\}+=\{DB\}\\
		Definimos otras dos nuevas relaciones\\
		U(D,B) con D $\rightarrow$ B\\
		V(D,A,C) con DAC $\rightarrow$ DAC\\
		En esta nueva partición se pierde AB $\rightarrow$ CD.\\
		Observamos que en U y V ya no se tienen violaciones, por lo tanto el esquema en BCNF para R es:\\
		\textbf{T(A,B,E)} con \textbf{ABE $\rightarrow$ ABE}\\
		\textbf{U(D,B)} con \textbf{D $\rightarrow$ B}\\
		\textbf{V(D,A,C)} con \textbf{DAC $\rightarrow$ DAC}\\
		
		\item Ya esta resuelto solo falta escribirlo aqui
		
	\end{enumerate}
\end{itemize}
   
   \item Para  cada una  de  las  siguientes  relaciones  con  su  respectivo  conjunto  de  dependencias funcionales:
   \begin{enumerate}
   	\item \textbf{R(A,B,C,D,E,F)} con \textbf{F = \{B $\rightarrow$ D, B $\rightarrow$ E, D $\rightarrow$ F, AB $\rightarrow$ C\}}
   	\item \textbf{R(A,B,C,D,E)} con \textbf{F = \{A $\rightarrow$ BC, B $\rightarrow$ D, CD $\rightarrow$ E, E $\rightarrow$ A\}}
   \end{enumerate}

   \begin{itemize}
   	\item Indica \textbf{todas las violaciones } a la \textbf{3NF}\\
   	Que no aparezca una llave candidata en el lazo izquierdo de las DF o que no aparezca a la derecha.\\
   	Calculamos la cerradura:\\
   	\begin{enumerate}
   		\item \{B\}+=\{BDE\}, \{D\}+=\{DF\}, \{AB\}+=\{ABCDE\} una llave para R es \textbf{AB}\\ Las dependencias que violan la 3NF son  B $\rightarrow$ D, B $\rightarrow$ E, D $\rightarrow$ F
   	    \item \{A\}+=\{ABCDE\}, \{B\}+=\{B\},\{CD\}+=\{CDE\}, \{E\}+=\{EABCD\} una llave para R es \textbf{A} ó \textbf{E}\\ B $\rightarrow$ D viola la 3NF
   	\end{enumerate}
   	\item \textbf{Normaliza} de acuerdo a la \textbf{3NF}\\
   	Para normalizar en 3NF se deben buscar superfluos por la izquierda y por la derecha.
   	\begin{enumerate}
   		\item \begin{itemize}
   			\item Superfluos por la izquierda:\\
   			En este caso la dependencia que tiene mas de un atributo por la izquierda el la DF que contiene a la llave del lado izquierdo por lo tanto no es necesario verificarlo.\\
   			\item Superflos por la derecha:\\
   			Haciendo uso de la propiedad de la union tenemos que F queda como F=\{B $\rightarrow$ DE, D $\rightarrow$ F, AB $\rightarrow$ C\}\\
   			Tomamos la dependencia que se violacion a 3NF y tenga mas de un atributo a la derecha, B $\rightarrow$ DE y buscamos elementos superfluos:\\
   			¿D es superfluo? B $\rightarrow$ E\\
   			obtenemos un nuevo conjunto de dependencias funcionales F' = \{B $\rightarrow$ E, D $\rightarrow$ F, AB $\rightarrow$ C\} y calculamos la cerradura para B\\
   			\{B\}+=\{BE\} como D no aparece por lo tanto D no es superfluo.\\
   			
   			¿E es superfluo? B $\rightarrow$ D\\
   			obtenemos un nuevo conjunto de dependencias funcionales F' = \{B $\rightarrow$ D, D $\rightarrow$ F, AB $\rightarrow$ C\} y calculamos la cerradura para B\\
   			\{B\}+=\{BDF\} como E no aparece por lo tanto E no es superfluo.\\
   			
   			Así que obtenemos que F$_{min}$ = \{B $\rightarrow$ DE, D $\rightarrow$ F, AB $\rightarrow$ C\} a partir de este conjunto creamos una relación por cada DF\\
   			R$_1$(B,D,E) con B $\rightarrow$ DE\\
   			R$_2$(D,F) con  D $\rightarrow$ F\\
   			R$_3$(A,B,C) con AB $\rightarrow$ C\\
   			
   			Como la llave esta contenida en la relación R$_3$ por lo tanto esta es la normalización para R en 3NF.\\
   			\item Resuelto solo falta escribirlo aqui.
   			
   		\end{itemize}
   	\end{enumerate}
   	
   \end{itemize}
 
    	\item Sea el esquema:\\
    	\begin{center}
    		\textbf{R(A,B,C,D,E,F)} con \textbf{F=\{BD $\rightarrow$ E, CD $\rightarrow$ A, E $\rightarrow$ C, B $\rightarrow$ D\}}
    	\end{center} 
    	\begin{itemize}
    		\item ¿Qué puedes decir de \textbf{{A}+} y \textbf{{F}+}?
    		\item Calcula \textbf{{B}+}, ¿qué puedes decir de esta cerradura?\\
    		\{B\}+=\{BDECA\} Casí contiene todos los atributos de la relación por lo tanto podemos agregarle el atributo F y sería una llave para la relación R.
    		\item  Obtén todas las \textbf{llaves candidatas}.\\
    		\{BD\}+=\{BDECA\},\{CD\}+=\{CDA\}, \{E\}+=\{EC\}, \{B\}+=\{BDECA\}\\
    		Las llaves candidatas son: BF , BDF
    		\item  ¿R cumple con \textbf{BCNF}? ¿Cumple con \textbf{3NF}? (en caso contrario normaliza) \\
    		Ya esta resuelto
    		\item  Se ha decidido dividir \textbf{R} en las siguientes relaciones \textbf{S(A,B,C,D,F)} y \textbf{T(C,E)}, ¿se puede recuperar la información de \textbf{R}? No podemos recuperar toda la informacion de R
    	\end{itemize}
    
    
    	\item Para  cada  uno  de  los  esquemas,  con  su  respectivo  conjunto  de  dependencias  multivaluadas, resuelve los siguientes puntos:
    	\begin{enumerate}
    		\item \textbf{R(A,B,C,D)} con \textbf{DMV = {AB $\twoheadrightarrow$ C, B $\rightarrow$ D}}
    		\item \textbf{R(A,B,C,D,E)} con \textbf{DMV = { A $\twoheadrightarrow$ B, AB $\rightarrow$ C, A $\rightarrow$ D, AB $\rightarrow$ E}}
    	\end{enumerate}
    
    \begin{itemize}
    	\item Encuentra \textbf{todas las violaciones} a la \textbf{4NF}
    	\item \textbf{Normaliza} de acuerdo a la \textbf{4NF}
    \end{itemize}
    
    	\item Se tiene la siguiente relación:   	
    	\begin{center}
    		 \textbf{R(idEnfermo, idCirujano, fechaCirugía, nombreEnfermo, direcciónEnfermo, nombreCirujano,n nombreCirugía, medicinaSuministrada, efectosSecundarios)}
    	\end{center}	 
    	
    	\begin{itemize}
    	\item Expresa las siguientes restricciones en forma de \textbf{dependencias funcionales}:\\
    	A un enfermo sólo se le da una medicina después de la operación. Si existen efectos secundarios estos dependen sólo de la medicina suministrada. Sólo puede existir un efecto secundario por medicamento.
    	\item Especifica  otras \textbf{dependencias  funcionales} o \textbf{multivaluadas} que  deban  satisfacerse  en  la relación R. Por cada una que definas, deberá aparecer un enunciado en español como en el inciso anterior.
    	\item \textbf{Normaliza} utilizando el conjunto de dependencias establecido en los puntos anteriores.
    \end{itemize}
    	
\end{enumerate}
\end{document}