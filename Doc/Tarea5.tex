\documentclass{article}
\usepackage[utf8]{inputenc}
\usepackage[spanish]{babel}
\usepackage{amssymb}
\usepackage{amsmath}

% Formato de página
\usepackage[letterpaper, margin = 1.5cm]{geometry}

% Más opciones para enumerar
\usepackage{enumitem}

% Manejo de imágenes
\usepackage{graphicx}
\usepackage{wrapfig}
\graphicspath{{img/}}
\usepackage{float}

\begin{document}
    \title{
        Fundamentos de bases de datos \\
        Tarea 5 \\
        Dependencias y Normalización
    }
    \author{
        Díaz Gómez Silvia \\
        Eugenio Aceves Narciso Isaac \\
        Quiroz Castañeda Edgar
    }
    \date {
        26 de Abril del 2019    
    }
    \maketitle
    
    \begin{enumerate}
    	
    
    \item { \textbf{Preguntas de repaso}}
    \begin{enumerate}[label = \alph*.]
        \item ¿Qué es una dependencia funcional y cómo se define?\\
        Es una relación entre un conjunto de atributos $X$ y
        otro $Y$, denotada como $X \rightarrow Y$, donde a cada
        posible valor de $Y$ se le asocia un único valor de $Y$.
        También se les llama propiedades semánticas.
        Por otra parte, se crean a partir de las '\textit{reglas del negocio}' y 
        características concidas de las relaciones (semántica) ó reglas de inferencia.
        
        \item ¿Para qué sirve el concepto de \textbf{dependencia} en la normalización?
        Sirve para especificar restricciones entre los atributos
        de la relación, y determinar si son legales bajo esas
        restricciones. Además es en base a estas dependencias
        funcionales que se define el concepto de llave y nos permite 
        identificar tablas que potencialmente podrían introducir anomalías
        (redundancia o pérdida de integridad) que es de donde vienen
        las formas normales que se y la normalización de relaciones.
        
        \item Sea A la llave de R(A, B, C). Indica \textbf{todas} las dependencias funcionales que implica \textbf{A}. \\
        $\{A \rightarrow A$, $A \rightarrow B$, 
        $A \rightarrow C$, $A \rightarrow AB$,
        $A \rightarrow AC$,$A \rightarrow BC$, 
        $A \rightarrow ABC\}$\\
        
        \item ¿Qué es una forma normal? ¿Cuál es el objetivo de normalizar un modelo de datos? \\
        Una relación está en forma normal si cumple una serie
        de restricciones respecto a su esquema.\\
        Cada regla aumenta el grado de normalización.\\
        El propósito de esto es eliminar redundancia y 
        garantizar ``JOIN`` sin pérdida, todo esto sin perder
        dependencias funcionales sen el proceso.
        \item ¿En qué casos es preferible lograr \textbf{3NF} en vez de \textbf{BCNF}?\\
        La \textbf{BCNF} elimina toda la redundancia de la base
        de datos, pero se pueden perder dependencias funcionales
        en el proceso. En un caso real, si se pierden las
        dependencias funcionales entonces la base de datos ya no
        estaría modelando fielmente la realidad del caso de uso,
        por lo que no es aceptable perderlas.\\
        La \textbf{3NF} no elimina la redundancia completamente,
        pero mantiene todas las dependencias funcionales.\\
        Entonces, si al usar \textbf{BCNF} se pierden
        dependencias funcionales, es necesarios usar una forma
        de normalización menos estricta, entiéndase 
        \textbf{3NF}.
    \end{enumerate}

    \item
    Proporciona algunos ejemplos que demuestren que las siguientes reglas no son válidas:
    \begin{enumerate}
    	\item Si \textbf{A $\rightarrow$ B}, entonces  \textbf{B $\rightarrow$ A}\\
    	En un sistema de registro de vehiculos y conductores (particulares), si hay un auto, es necesario que le
    	asigne un propietario (quien pagará multas y se hará responsable) y entonces a todo auto 'A' le asigno una
    	persona con licencia 'B'. No obstante, no todas las personas con licencia estan obligadas a tener un auto 
    	propio y entonces la segunda dependencia funcional realmente no existe/se cumple.
    	
    	\item Si \textbf{AB $\rightarrow$ C}, entonces  \textbf{A $\rightarrow$ C y B $\rightarrow$ C}\\
    	Podemos tomar 'A' como el dinero que tiene una persona, 'B' como el número de lugares disponibles en
    	sus cines asociados (digamos, los más cercanos de la franquicia Mexipolis) y 'C' un atributo que nos
    	indica si la persona ha visto o no la película \textit{"DC:Los Vendedores, Jugada Final"} porque queremos
    	saber si podemos hablar de \textit{\textbf{spoilers}} con la persona.
    	
    	La primera regla tiene sentido (El que tengas dinero suficiente para un boleto y que haya lugares en el cine
    	pueden determinar si vas a ver la película/ya la viste) pero ni A $\rightarrow$ C se cumple, ya que el cine 
    	podría estar lleno y no importaría que tienes dinero, ni B $\rightarrow$ C, ya que puede haber lugares pero
    	sin dinero para boletos no puedes ir al cine. Por lo tanto ninguna de las dos puede determinar realmente si
    	ya viste la pelicula (a ambas les falta información para eso) y no existen esas dependencias funcionales.
    	
    	\item Si \textbf{A $\twoheadrightarrow$ C}, entonces \textbf{A $\rightarrow$ C}\\
    	Digamos que 'A' es el identificador de un equipo de tareas en FBD y 'C' es la calificación de una tarea. Entonces
    	podemos tener una tabla con (1,5T1),(1,6T2),(2,10T2) que representa que el equipo 1 ha entregado dos tareas con 5 
    	y 6 de calificación mientras que el equipo 2 sólo entregó la tarea 2 pero sacó 10 en ella.
    	
    	La primera representa una DMV (Dependencia Multivaluada). Si se cumpliera lo segundo ($\rightarrow$), estaríamos 
    	obligando a que a un equipo siempre se le asigne la misma claifiación/tarea, i.e. (1,10T2),(1,7T3) no sería válido
    	porque a cada valor de 'A' le debe corresponder un único valor en 'C', pero esto no es lo que se planteaba la
    	primera de las reglas originalmente.
    	
    \end{enumerate}
       
    \item Para cada uno de los esquemas que se muestran a continuación:
    
    \begin{enumerate}
    	\item \textbf{R(A,B,C,D,E)} con \textbf{F=\{AB $\rightarrow$ CD, E $\rightarrow$ C, D $\rightarrow$ B \}}
    	\item \textbf{R(A,B,C,D,E)}  con \textbf{F=\{AB $\rightarrow$ C, DE $\rightarrow$ C, B $\rightarrow$ D \}}
    \end{enumerate}   

\begin{itemize}
	\item Especifica  de ser posible \textbf{dos DF no triviales} que se puedan derivar de las dependencias funcionales dadas.
	Usando la cerradura de atributos.
	\begin{enumerate}
		\item \textbf{\{AD\}}+= \{ADBC\} y \textbf{\{AE\}}+=\{AEC\} a partir de estas cerraduras obtenemos las siguientes dependencias: AD $\rightarrow$ BC y AE $\rightarrow$ C\\ 
		\item \textbf{\{AB\}}+= \{ABCD\} y \textbf{\{BE\}}+=\{EBDC\} a partir de estas cerraduras obtenemos las siguientes dependencias: AB $\rightarrow$ CD y BE $\rightarrow$ DC\\ 
		 
		
		
	\end{enumerate}
	\item Indica una \textbf{llave candidata} para \textbf{R} \\
	Usamos la cerradura para encontrar una llave candidata
	\begin{enumerate}
		\item \{AB\}+= \{ABCD\}, \{E\}+= \{EC\}, \{D\}+= \{DB\}, la cerradura de AB es la que contiene mas elementos de la relación por lo tanto una llave para R sería : \textbf{ABE}\\ 
		\item \{AB\}+= \{ABCD\}, \{DE\}+= \{E\}, \{B\}+= \{BD\}, la cerradura de AB es la que contiene mas elementos de la relación por lo tanto una llave para R sería : \textbf{ABE}\\ 
		
		
		
	\end{enumerate}
	
	\item Especifica \textbf{todas las violaciones}  a la \textbf{BCNF}\\
	\begin{enumerate}
		\item Las tres dependencias son violaciones a BCNF porque no aparece del lado izquierdo de las DF la llave que es ABE.\\
		
		\item Las tres dependencias son violaciones a BCNF porque no aparece del lado izquierdo de las DF la llave que es ABE.
	\end{enumerate}
	\item \textbf{Normaliza} de acuerdo a \textbf{BCNF}, asegúrate de indicar cuáles son las relaciones resultantes con sus respectivas dependencias funcionales.
	\begin{enumerate}
		\item Como todas las DF son violaciones, tomamos a AB $\rightarrow$ CD y su cerradura es  \{AB\}+= \{ABCD\}.
		Así que definimos dos nuevas relaciones S y T,\\
		S(A,B,C,D) con \{AB $\rightarrow$ CD , D $\rightarrow$ B\}\\
		T(A,B,E) con ABE $\rightarrow$ ABE y perdemos E $\rightarrow$ C\\
		En S la llave es AB por lo tanto D $\rightarrow$ B es una violación.\\
		Ahora para S tomamos D $\rightarrow$ B y calculamos la cerradura para \{D\}+=\{DB\}\\
		Definimos otras dos nuevas relaciones\\
		U(D,B) con D $\rightarrow$ B\\
		V(D,A,C) con DAC $\rightarrow$ DAC\\
		En esta nueva partición se pierde AB $\rightarrow$ CD.\\
		Observamos que en U y V ya no se tienen violaciones, por lo tanto el esquema en BCNF para R es:\\
		\textbf{T(A,B,E)} con \textbf{ABE $\rightarrow$ ABE}\\
		\textbf{U(D,B)} con \textbf{D $\rightarrow$ B}\\
		\textbf{V(D,A,C)} con \textbf{DAC $\rightarrow$ DAC}\\
		
		\item Como todas las DF son violaciones, tomamos a B $\rightarrow$ D y su cerradura es  \{B\}+= \{BD\}.
		Así que definimos dos nuevas relaciones S y T,\\
		S(B,D) con \{B $\rightarrow$ D\}\\
		T(B,A,C,E) con AB $\rightarrow$ C y perdemos DE $\rightarrow$ C\\
		En S, B es llave por lo tanto  S ya esta en BCNF.\\
		Ahora en T la llave sigue siendo ABE por lo tanto  AB $\rightarrow$ C es violación, calculamos la cerradura para \{AB\}+=\{ABC\}\\
		Definimos otras dos nuevas relaciones\\
		U(A,B,C) con AB $\rightarrow$ C\\
		V(A,B,E) con ABE $\rightarrow$ ABE\\
		Observamos que en U y V ya no se tienen violaciones, por lo tanto el esquema en BCNF para R es:\\
		\textbf{S(B,D)} con \textbf{B $\rightarrow$ D}\\
		\textbf{U(A,B,C)} con \textbf{AB $\rightarrow$ C}\\
		\textbf{V(A,B,E)} con \textbf{ABE $\rightarrow$ ABE}\\ 
		
	\end{enumerate}
\end{itemize}
   
   \item Para  cada una  de  las  siguientes  relaciones  con  su  respectivo  conjunto  de  dependencias funcionales:
   \begin{enumerate}
   	\item \textbf{R(A,B,C,D,E,F)} con \textbf{F = \{B $\rightarrow$ D, B $\rightarrow$ E, D $\rightarrow$ F, AB $\rightarrow$ C\}}
   	\item \textbf{R(A,B,C,D,E)} con \textbf{F = \{A $\rightarrow$ BC, B $\rightarrow$ D, CD $\rightarrow$ E, E $\rightarrow$ A\}}
   \end{enumerate}

   \begin{itemize}
   	\item Indica \textbf{todas las violaciones } a la \textbf{3NF}\\
   	Que no aparezca una llave candidata en el lazo izquierdo de las DF o que no aparezca a la derecha.\\
   	Calculamos la cerradura:\\
   	\begin{enumerate}
   		\item \{B\}+=\{BDE\}, \{D\}+=\{DF\}, \{AB\}+=\{ABCDE\} una llave para R es \textbf{AB}\\ Las dependencias que violan la 3NF son  B $\rightarrow$ D, B $\rightarrow$ E, D $\rightarrow$ F
   	    \item \{A\}+=\{ABCDE\}, \{B\}+=\{B\},\{CD\}+=\{CDEAB\}, \{E\}+=\{EABCD\} una llave para R es \textbf{A}, \textbf{E} o \textbf{DC}\\ B $\rightarrow$ D viola la 3NF
   	\end{enumerate}
   	\item \textbf{Normaliza} de acuerdo a la \textbf{3NF}\\
   	Para normalizar en 3NF se deben buscar superfluos por la izquierda y por la derecha.
   	\begin{enumerate}
   		\item \begin{itemize}
   			\item Superfluos por la izquierda:\\
   			En este caso la dependencia que tiene mas de un atributo por la izquierda es la DF que contiene a la llave del lado izquierdo por lo tanto no es necesario verificarlo.\\
   			\item Superflos por la derecha:\\
   			Haciendo uso de la propiedad de la unión tenemos que F queda como F=\{B $\rightarrow$ DE, D $\rightarrow$ F, AB $\rightarrow$ C\}\\
   			Tomamos la dependencia que se violacion a 3NF y tenga mas de un atributo a la derecha, B $\rightarrow$ DE y buscamos elementos superfluos:\\
   			¿D es superfluo? B $\rightarrow$ E\\
   			obtenemos un nuevo conjunto de dependencias funcionales F' = \{B $\rightarrow$ E, D $\rightarrow$ F, AB $\rightarrow$ C\} y calculamos la cerradura para B\\
   			\{B\}+=\{BE\} como D no aparece por lo tanto D no es superfluo.\\
   			
   			¿E es superfluo? B $\rightarrow$ D\\
   			obtenemos un nuevo conjunto de dependencias funcionales F' = \{B $\rightarrow$ D, D $\rightarrow$ F, AB $\rightarrow$ C\} y calculamos la cerradura para B\\
   			\{B\}+=\{BDF\} como E no aparece por lo tanto E no es superfluo.\\
   			
   			Así que obtenemos que F$_{min}$ = \{B $\rightarrow$ DE, D $\rightarrow$ F, AB $\rightarrow$ C\} a partir de este conjunto creamos una relación por cada DF\\
   			R$_1$(B,D,E) con B $\rightarrow$ DE\\
   			R$_2$(D,F) con  D $\rightarrow$ F\\
   			R$_3$(A,B,C) con AB $\rightarrow$ C\\
   			
   			Como la llave esta contenida en la relación R$_3$ por lo tanto esta es la normalización para R en 3NF.\\
   			\end{itemize}
   			
   		\item \begin{itemize}	
   			\item Superfluos por la izquierda:\\
   			En este caso la dependencia que tiene mas de un atributo por la izquierda es CD $\rightarrow$ E y verificamos si algun atributo es superfluo.\\
   			¿C es superfluo? D $\rightarrow$ E\\
   			\textbf{\{D\}}+=\{D\}, E no aparece en la cerradura de D, por lo tanto C no es superfluo.\\
   			¿D es superfluo? C $\rightarrow$ E\\
   			\textbf{\{C\}}+=\{C\}, E no aparece en la cerradura de C, por lo tanto D no es superfluo.\\
   			Esta parte no era necesario verificarlo porque CD es llave candidata.
   			
   			\item Superflos por la derecha:\\
   			Tomamos la dependencia que tenga mas de un atributo a la derecha, A $\rightarrow$ BC y buscamos elementos superfluos:\\
   			¿B es superfluo? A $\rightarrow$ C \\
   			obtenemos un nuevo conjunto de dependencias funcionales F' = \{A $\rightarrow$ C, B $\rightarrow$ D, CD $\rightarrow$ E,E $\rightarrow$ A\} y calculamos la cerradura para A\\
   			\{A\}+=\{AC\} como B no aparece por lo tanto B no es superfluo.\\
   			
   			¿C es superfluo? A $\rightarrow$ B\\
   			obtenemos un nuevo conjunto de dependencias funcionales F' = \{A $\rightarrow$ B, B $\rightarrow$ D, CD $\rightarrow$ E,E $\rightarrow$ A\} y calculamos la cerradura para A\\
   			\{A\}+=\{ABD\} como C no aparece por lo tanto C no es superfluo.\\
   			
   			Así que obtenemos que \textbf{F$_{min}$}= \{A $\rightarrow$ BC, B $\rightarrow$ D, CD $\rightarrow$ E, E $\rightarrow$ A\} a partir de este conjunto creamos una relación por cada DF\\
   			R$_1$(A,B,C) con A $\rightarrow$ BC\\
   			R$_2$(B,D) con  B $\rightarrow$ D\\
   			R$_3$(C,D,E) con CD $\rightarrow$ E\\
   			R$_4$(E,A) con E $\rightarrow$ A\\
   			
   			Como la llave esta contenida en la relación R$_3$ por lo tanto esta es la normalización para R en 3NF.\\
   			
   		\end{itemize}
   	\end{enumerate}
   	
   \end{itemize}
 
    	\item Sea el esquema:\\
    	\begin{center}
    		\textbf{R(A,B,C,D,E,F)} con \textbf{F=\{BD $\rightarrow$ E, CD $\rightarrow$ A, E $\rightarrow$ C, B $\rightarrow$ D\}}
    	\end{center} 
    	\begin{itemize}
    		\item ¿Qué puedes decir de \textbf{{A}+} y \textbf{{F}+}?\\
    		\textbf{{A}+}= \{A\} y \textbf{{F}+}=\{F\} No alcanzan ningun otro atributo.
    		\item Calcula \textbf{{B}+}, ¿qué puedes decir de esta cerradura?\\
    		\{B\}+=\{BDECA\} Casí contiene todos los atributos de la relación por lo tanto podemos agregarle el atributo F y sería una llave para la relación R.
    		\item  Obtén todas las \textbf{llaves candidatas}.\\
    		\{BD\}+=\{BDECA\},\{CD\}+=\{CDA\}, \{E\}+=\{EC\}, \{B\}+=\{BDECA\}\\
    		Las llaves candidatas son: BF.
    		Una llave debe ser mínima, y como BF es llave
    		ninguna conjunto de tres o más atributos puede ser
    		llave en este esquema. Luego, todas los de conjuntos
    		de un sólo atributo no son llaves, pues ninguna
    		dependencia alcanza a F, por lo que para que
    		alcance a todos, debe tener explícitamente a F.\\
    		Luego, FA, FC y FD sólo tiene dependencias triviales
    		, pues no forman parte de ninguna dependencia.\\
    		Y $\{EF\}+ = EFC$, por lo que no es llave.\\
    		Entonces la única llave es BF.
    		\item  ¿R cumple con \textbf{BCNF}? ¿Cumple con \textbf{3NF}? (en caso contrario normaliza) \\
    		R no cumple BCNF ni con 3NF.\\
    		La llave candidata es BF.
    		\begin{itemize}
    			\item Normalización con BCNF\\
    			Todas las dependencias son violaciones a BCNF, tomamos BD $\rightarrow$ E, así tenemos la partición como:\\
    			S(B,D,E) con BD $\rightarrow$ E, (ya no tiene ninguna violación)\\
    			T(B,D,A,C,F) con \{ B $\rightarrow$ D, CD $\rightarrow$ A\}\\
    			y E $\rightarrow$ C se pierde.
    			en T las DF son violaciones por lo tanto hacemos una nueva partición,\\
    			U(B,D) con B $\rightarrow$ D, (ya no tiene ninguna violación) \\
    			V(B,A,C,F) con BACF $\rightarrow$ BACF\\
    			y perdemos CD $\rightarrow$ A.\\
    			
    			Nuestro esquema normalizado con BCNF queda de la siguiente manera:\\
    			S(B,D,E) con BD $\rightarrow$ E,\\
    			U(B,D) con B $\rightarrow$ D,  \\
    			V(B,A,C,F) con BACF $\rightarrow$ BACF
    			
    			\item Normalización con 3NF\\
    			Todas las dependecias son violaciones a 3NF\\
    			\begin{itemize}
    				\item Superfluos por la izquierda\\
    				Tomamos a BD $\rightarrow$ E,\\
    				¿B? D $\rightarrow$ E,\\
    				\textbf{\{D\}}+=\{D\}, E no aparece en la cerradura de D, por lo tanto B no es superfluo.
    				¿D? B $\rightarrow$ E,\\
    				\textbf{\{B\}}+=\{BDECA\}, E aparece en la cerradura de B, por lo tanto D es superfluo.\\
    				
    				Entonces \textbf{F$_{min}$}= \{B $\rightarrow$ E, CD $\rightarrow$ A, E $\rightarrow$ C, B $\rightarrow$ D\} y por la propiedad de la unión \textbf{F$_{min}$}= \{B $\rightarrow$ DE, CD $\rightarrow$ A, E $\rightarrow$ C\} \\
    				
    				
    				\item Superfluos por la derecha:\\
    				Tomamos la dependencia que tenga mas de un atributo a la derecha, en este caso es B $\rightarrow$ DE y buscamos elementos superfluos:\\
    				¿D es superfluo? B $\rightarrow$ E \\
    				obtenemos un nuevo conjunto de dependencias funcionales F' = \{B $\rightarrow$ E, CD $\rightarrow$ A, E $\rightarrow$ C\} y calculamos la cerradura para B\\
    				\{B\}+=\{BEC\} como D no aparece por lo tanto D no es superfluo.\\
    				
    				¿E es superfluo? B $\rightarrow$ D \\
    				obtenemos un nuevo conjunto de dependencias funcionales F' = \{B $\rightarrow$ D, CD $\rightarrow$ A, E $\rightarrow$ C\} y calculamos la cerradura para B\\
    				\{B\}+=\{BD\} como E no aparece por lo tanto E no es superfluo.\\
    				
    				Así que obtenemos que \textbf{F$_{min}$}= \{B $\rightarrow$ DE, CD $\rightarrow$ A, E $\rightarrow$ C\} a partir de este conjunto creamos una relación por cada DF\\
    				R$_1$(B,D,E) con B $\rightarrow$ DE\\
    				R$_2$(C,D,A) con  CD $\rightarrow$ A\\
    				R$_3$(E,C) con E $\rightarrow$ C\\
    				
    				Y como la llave no esta en ninguna relación creamos una nueva relación que la contenga R$_4$(B,F) con BF $\rightarrow$ BF.\\
    				
    				Por lo tanto R$_1$, R$_2$, R$_3$ y R$_4$ es la normalización en 3NF.
    			\end{itemize}
    			
    		\end{itemize}
    		\item  Se ha decidido dividir \textbf{R} en las siguientes relaciones \textbf{S(A,B,C,D,F)} y \textbf{T(C,E)}, ¿se puede recuperar la información de \textbf{R}? No podemos recuperar toda la informacion de R porque se pierde la dependencia BD $\rightarrow$ E.
    	\end{itemize}
    
    
    	\item Para  cada  uno  de  los  esquemas,  con  su  respectivo  conjunto  de  dependencias  multivaluadas, resuelve los siguientes puntos:
    	\begin{enumerate}
    		\item \textbf{R(A,B,C,D)} con \textbf{DMV = {AB $\twoheadrightarrow$ C, B $\rightarrow$ D}}
    		\item \textbf{R(A,B,C,D,E)} con \textbf{DMV = { A $\twoheadrightarrow$ B, AB $\rightarrow$ C, A $\rightarrow$ D, AB $\rightarrow$ E}}
    	\end{enumerate}
    
    \begin{itemize}
    	\item Encuentra \textbf{todas las violaciones} a la \textbf{4NF}
    	 \begin{enumerate}
    	     \item \textbf{R(A,B,C,D)} con \textbf{DMV = {AB $\twoheadrightarrow$ C, B $\rightarrow$ D}}\\
    	     Consideremos las posibles llaves calculando las cerraduras.
    	     \[\{B\}+ = \{BD\}\]
    	     Por lo que una llave puede ser \textbf{ABC}.\\
    	     Como ninguna de las dependencias tiene a esta llave en su
    	     parte izquierda, entonces toda \textbf{DMV} son violaciones
    	     a la cuarta
    	     forma normal.
    	     \item  \textbf{R(A,B,C,D,E)} con \textbf{DMV = { A $\twoheadrightarrow$ B, AB $\rightarrow$ C, A $\rightarrow$ D, AB $\rightarrow$ E}} \\
    	     Consideremos las posibles llaves calculando las cerraduras.
    	     \begin{align*}
    	         \{AB\}+ &= \{ABCDE\} \\
    	         \{A\}+ &= \{AD\} \\
    	     \end{align*}
    	     Por lo que una llave es \textbf{AB}.\\
    	     Por lo que las violaciones son 
    	     \textbf{ A$\twoheadrightarrow$ B} y \textbf{A $\rightarrow$ D}
    	 \end{enumerate}
    	\item \textbf{Normaliza} de acuerdo a la \textbf{4NF}
    	\begin{enumerate}
    	    \item \textbf{R(A,B,C,D)} con \textbf{DMV = {AB $\twoheadrightarrow$ C, B $\rightarrow$ D}}\\
    	    Para intentar preservar las dependencias multivaluadas,
    	    primero empezemos seleccionando una violación que sea
    	    dependencia funcional, que es este caso sólo puede ser 
    	    \textbf{B $\rightarrow$ D}.
    	    Entonces, \textbf{R} se parte en dos nuevas tablas
    	    $R_1(B, D)$ y $R_2(A, B, C)$.\\
    	    En $R_1$ se preserva únicamente la dependencia funcional 
    	    \textbf{B $\rightarrow$ D}, que como incluye a todos los
    	    atributos de la relación es trivial, por lo que $R_1$ ya
    	    está en \text{4NF}.
    	    En $R_2$, sólo se preserva una dependencia multivaluada, pero
    	    esta incluye a todos los elementos de $R_2$, por lo que no es
    	    una violación a la \textbf{4NF}. Por lo que $R_2$ también ya
    	    está normalizada.\\
    	    Ahora, hay que revisar si la llave está contenida en alguna
    	    de las relaciones. Efectivamente, la llave está en $R_2$, por
    	    lo que no es necesario agregar ninguna relación.\\
    	    Entonces $R_1$ y $R_2$ son la normalización en \textbf{4NF} de
    	    \textbf{R}.
    	    \item \textbf{R(A,B,C,D,E)} con \textbf{DMV = { A $\twoheadrightarrow$ B, AB $\rightarrow$ C, A $\rightarrow$ D, AB $\rightarrow$ E}}\\
    	    Para intentar preservar las dependencias multivaluadas,
    	    primero empezamos seleccionando una violación que sea
    	    dependencia funcional, que es este caso sólo puede ser 
    	    \textbf{A $\rightarrow$ D}.\\
    	    Entonces, \textbf{R} se parte en dos nuevas tablas
    	    $R_1(A, D)$ y $R_2(A, B, C, E)$.\\
    	    En $R_1$ se preserva únicamente la dependencia funcional 
    	    \textbf{A $\rightarrow$ D}, que como incluye a todos los
    	    atributos de la relación es trivial, por lo que $R_1$ ya
    	    está en \text{4NF}. \\
    	    Mientras que en $R_2$ se siguen manteniendo todas las demás 
    	    dependencias multivaluadas.\\
    	    Por esto, se sigue teniendo que \textbf{AB} sigue siendo llave
    	    y por lo tanto \text{A $\twoheadrightarrow$ B} sigue siendo
    	    violación de \textbf{4NF}, y de hecho la única.\\
    	    Entonces, tomándola como violación, $R_2$ se parte en dos
    	    nuevas tablas $R_3(A, B)$ y $R_4(A, C, E)$.\\
    	    En $R_3$ se preserva únicamente \text{A $\twoheadrightarrow$ B},
    	    que como tiene a todos los atributos de la relación. no es 
    	    violación de la \textbf{4NF}, por lo que $R_3$ ya está
    	    normalizada.\\
    	    En $R_4$ no se preserva ninguna dependencia, por lo que las
    	    únicas presentes son las triviales, por lo que $R_4$ también
    	    ya está en \textbf{4NF}.\\
    	    Notemos que en este último paso se perdieron las dependencias
    	    funcionales \textbf{AB $\rightarrow$ E} y \textbf{AB
    	    $\rightarrow$ C}.
    	    Por último, como la llave original está contenida en $R_3$, 
    	    no es necesario agregar ninguna relación.\\
    	    Por lo que $R_1$, $R_3$ y $R_4$ son la normalización en
    	    \textbf{4NF} de \text{R}.
    	    
    	\end{enumerate}
    \end{itemize}
    
    	\item Se tiene la siguiente relación:   	
    	\begin{center}
    		 \textbf{R(idEnfermo, idCirujano, fechaCirugía, nombreEnfermo, direcciónEnfermo, nombreCirujano,n nombreCirugía, medicinaSuministrada, efectosSecundarios)}
    	\end{center}	 
    	
    \begin{itemize}
    	\item Expresa las siguientes restricciones en forma de \textbf{dependencias funcionales}:\\
    	\textit{A un enfermo sólo se le da una medicina después de la operación. Si existen efectos secundarios estos dependen sólo de 
    	la medicina suministrada. Sólo puede existir un efecto secundario por medicamento.}
    	
    	Sea \textbf{R}(A,B,C,D,E,F,G,H,I), cada letra el atributo que le corresponde de acuerdo a su posición y la definición de \textbf{R}.
    	\begin{enumerate}
    	    \item $A,C,G \rightarrow H$\\
    	    \item $H \rightarrow I$\\
    	\end{enumerate}
    	
    	\item Especifica  otras \textbf{dependencias  funcionales} o \textbf{multivaluadas} que  deban  satisfacerse  en  la relación R. 
    	Por cada una que definas, deberá aparecer un enunciado en español como en el inciso anterior.
    	\textit{Cada cirugía tiene exactamente a un cirujano registrado como a cargo de esta. Un cirujano puede haber estado a cargo de
    	varias cirujias distintas. Un enfermo puede haber tenido varias cirujias. En cada cirujía se trata a un solo enfermo.}
    	\begin{enumerate}
    	    \item $C,G \rightarrow B,F$\\
    	    \item B,F $\twoheadrightarrow$ C,G\\
    	    \item A $\twoheadrightarrow$ C,G\\
    	    \item C,G $\rightarrow$ A\\
    	\end{enumerate}
    	
    	\item \textbf{Normaliza} utilizando el conjunto de dependencias establecido en los puntos anteriores.
    \end{itemize}
    	
\end{enumerate}
\end{document}
